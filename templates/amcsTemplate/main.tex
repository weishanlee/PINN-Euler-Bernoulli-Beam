
\documentclass[12pt, reqno]{amsart}
\usepackage{amsmath, amsthm, amscd, amsfonts, amssymb, graphicx, xcolor}
\usepackage[bookmarksnumbered, colorlinks, plainpages]{hyperref}
\usepackage{geometry}
\geometry{a4paper, left=2.5cm, right=2.5cm, top=2.5cm, bottom=2.5cm}

\newtheorem{theorem}{Theorem}[section]
\newtheorem{lemma}[theorem]{Lemma}
\newtheorem{proposition}[theorem]{Proposition}
\newtheorem{corollary}[theorem]{Corollary}
\theoremstyle{definition}
\newtheorem{definition}[theorem]{Definition}
\newtheorem{example}[theorem]{Example}
\newtheorem{exercise}[theorem]{Exercise}
\newtheorem{conclusion}[theorem]{Conclusion}
\newtheorem{conjecture}[theorem]{Conjecture}
\newtheorem{criterion}[theorem]{Criterion}
\newtheorem{summary}[theorem]{Summary}
\newtheorem{axiom}[theorem]{Axiom}
\newtheorem{problem}[theorem]{Problem}
\theoremstyle{remark}
\newtheorem{remark}[theorem]{Remark}
\numberwithin{equation}{section}

\begin{document}
\setcounter{page}{1}

\color{darkgray}{
\noindent 
{\small Annals of Mathematics and Computer Science}\hfill     {\small ISSN: 2789-7206}\\
{\small Vol 20 (2024) 1-3}\hfill  {\small https://doi.org/10.56947/amcs.v20.223}}

\centerline{}

\centerline{}


%------------------------------------------------------------------------------

%Title of the paper
\title[Short Title]{Title of Paper}

%Author names and affiliations
\author[F. Author, S. Author]{First Author$^1$ and Second Author$^2$$^{*}$}

\address{$^{1}$ Department of Mathematics, University of California, San Diego, USA.}
\email{\textcolor[rgb]{0.00,0.00,0.84}{first@amcs.org}}

\address{$^{2}$ Department of Computer Science, Moscow State University, Moscow, Russia}
\email{\textcolor[rgb]{0.00,0.00,0.84}{second@ieee.org}}


%\dedicatory{This paper is dedicated to Professor ABCD}

\date{Received: xxxxxx; Revised: yyyyyy; Accepted: zzzzzz.
\newline \indent $^{*}$ Corresponding author
\newline \indent © The Author(s) 2025. This article is licensed under a Creative Commons Attribution-
\newline \indent NonCommercial-NoDerivatives 4.0
International License. To view a copy of the licence, visit 
\newline \indent \url{https://creativecommons.org/licenses/by-nc-nd/4.0/}}


%Abstract, keywords, math subject classification
\begin{abstract}
After the initial acceptance the author must prepare the manuscript according to the journal's format. Authors should use the following style file as a template. The abstract should be short and concise. Do not use complex mathematical expressions. Only Greek alphabet letters and upper/lower case text should be used.
\newline
\newline
\noindent \textit{Keywords.} Analysis, PDEs, machine learning, cybernetics
\newline
\noindent \textit{2020 Mathematics Subject Classification.} Primary 46L55; Secondary 44B20
\end{abstract} \maketitle



%------------------------------------------------------------------------------

\section{Introduction and preliminaries}

\noindent Here you should state the introduction, preliminaries and
your notation. Authors are required to state clearly the
contribution of the paper and its significance in the introduction.
There should be some survey of relevant literature.

\subsection{Instructions for author(s)}

Manuscripts should be typeset in English with double spacing by
using AMS-LaTex. The authors are encouraged to use the journal style
file that has been developed for LaTeX2e standard.

While you are preparing your paper, please take care of the
following:
\begin{enumerate}
\item Abstract: 150 words or less with no reference number therein.\\
\item MSC2020: Primary 1-2 items; and Secondary is optional. This is not required for articles in computer science.\\

\item Key words: At least 2 items and at most 5 items.\\
\item Authors: Full names, mailing addresses and emails of all authors.\\
\item Margins: A long formula should be broken into two or more lines. Empty spaces in the text should be removed.\\
\item Tags (Formula Numbers): Use $\backslash label\{A\}$ and $\backslash eqref\{A\}$. Remove unused tags. \\
\item Acknowledgement (optional): At the end of paper but preceding to References.\\
\item References: Use $\backslash cite\{MM\}$ to refer to the specific book/paper [2] (with $\backslash bibitem~\{MM\}$) in the text. Only cited references must be listed in the bibliography. References should be listed in the alphabetical order according to the surnames of the first author at the end of the paper using the APA style. Each reference must have the DOI.
References should be cited in the text as, e.g., [2] or [3, Theorem 4.2], etc.  \\
\item Abbreviations: Abbreviations of titles of periodicals/books should be given by using Math. Reviews, see Abbreviations of names of serials or MRLookup.

\end{enumerate}

\section{Main results}

Here is an example of a definition. 

\begin{definition} Let $A$ be a $C^*$-algebra. A mapping
$\phi :A\rightarrow \mathbb{C} $ is
called a positive linear functional on $A$ if it satisfies the following conditions:

\begin{enumerate}

\item $\phi(\alpha x+\beta y)=\alpha \phi(x)+\beta\phi(y)$ for all $\lambda ,\beta \in  \mathbb{C}$ and $x,y\in A$.

\item $\phi(x)\geq 0$ for all $a\geq 0$ in $A$.
\end{enumerate}
\end{definition}

%---------------------------------------------------------------------------------------%

Here is an example of a table.

\begin{table}[ht]
\caption{}\label{eqtable}
\renewcommand\arraystretch{1.5}
\noindent\[
\begin{array}{|c|c|c|}
\hline
1&2&3\\
\hline f(x)&g(x)&h(x)\\
\hline a&b&c\\
\hline
\end{array}
\]
\end{table}

Here is an example of a matrix.
\begin{equation*}
\begin{bmatrix}
1 & 2 \\
3 &  4
\end{bmatrix}
\end{equation*}


Here is an example.

%---------------------------------------------------------------------------------------%

\begin{example} Let $A$ be the $C^*$-algebra of $n\times n$ complex matrices. Define $Tr: A\rightarrow \mathbb{C}$ to be the canonical trace of a matrix. Then we have
\begin{equation}\label{2.1}
Tr(\alpha x+\beta y)=\alpha Tr(x)+\beta Tr(y).
\end{equation}
for all $\lambda ,\beta \in  \mathbb{C}$ and $x,y\in A$. It follows that $Tr$ is linear functional on $A$.
\end{example}

%---------------------------------------------------------------------------------------%

The following is an example of a theorem and a proof. Please note how to refer to a formula.

%---------------------------------------------------------------------------------------%

\begin{theorem}\label{theo1}
Let $G$ be a finite group acting on a second countable compact Hausdorff space $X$. Suppose that $\mu$ is a finite Borel measure on $X$. Then the induced bimodule $\mathcal{H}_\mu\times \mathcal{H}_\mu$ has almost central unit vectors.
\end{theorem}

%---------------------------------------------------------------------------------------%

\begin{proof}
Since $f$ is uniformly continuous on $X$ there exists $\delta>0$ such that $|f(x)-f(y)|<\epsilon$ for all $x,y\in X$ with $d(x,y)<\delta$. It follows that

\begin{align}\label{eq}
|f(xr)-f(yr)|&=|f(xr)-f(yr)|\nonumber\\
&=|f(xr)-f(yr)|\nonumber\\
&< \epsilon
\end{align}



for all $x,y\in E_n$, $r\in G$, $n>\frac{2}{\delta}$. Let $\bigtriangleup = \{(s, s): s\in G\}$ be the diagonal of $G\times G$. It follows from Equation~\eqref{eq} that  $||\pi(f)U(r)\zeta_n-\zeta_n\pi(f)U(r)||\rightarrow 0$ for all $f\in C(X)$, $r\in G$.
\end{proof}

%---------------------------------------------------------------------------------------%

The following is an example of a remark.

%---------------------------------------------------------------------------------------%

\begin{remark}
The purpose of this remark is to refer to the Theorem
\ref{theo1}. We also want to \cite{Brown, Connes}. 
\end{remark}

%---------------------------------------------------------------------------------------%

Again, note how we refer to Theorem \ref{theo1} and formula \eqref{2.1}.
\\
\\
{\bf Acknowledgement.} Acknowledgements could be placed at the end
of the text but precede the references.


\bibliographystyle{amsplain}
\begin{thebibliography}{99}

\bibitem{Bekka1} Bekka, B. (2006). Property (T) for C$^*$-algebras. Bulletin of the London Mathematical Society, 38(5), 857-867.
\url{https://doi.org/10.1112/S0024609306018765}

\bibitem{Bekka2}  Bekka, B., de La Harpe, P., \& Valette, A. (2008). Kazhdan's property (T). Cambridge university press.
\url{https://doi.org/10.1017/CBO9780511542749}

\bibitem{Brown} Brown, N. P. (2006). Kazhdan's property T and C$^*$-algebras. Journal of Functional Analysis, 240(1), 290-296.
\url{https://doi.org/10.1016/j.jfa.2006.05.003}


\bibitem{Connes}  Connes, A., \& Jones, V. (1985). Property T for von Neumann algebras. Bulletin of the London Mathematical Society, 17(1), 57-62.
\url{https://doi.org/10.1112/blms/17.1.57}

\bibitem{Green}  Green, P. (1978). The local structure of twisted covariance algebras. Acta Mathematica, 140, 191-250.
\url{https://doi.org/10.1007/BF02392308}
    
\bibitem{Kazhdan} Kazhdan, D. A. (1967). Connection of the dual space of a group with the structure of its close subgroups. Funktsional'nyi Analiz i ego Prilozheniya, 1(1), 71-74.
\url{https://doi.org/10.1007/BF01075866}

\bibitem{Leung} Leung, C. W., \& Ng, C. K. (2009). Property (T) and strong property (T) for unital C$^*$-algebras. Journal of Functional Analysis, 256(9), 3055-3070.
\url{https://doi.org/10.1016/j.jfa.2009.01.004}

\end{thebibliography}

\end{document}

%------------------------------------------------------------------------------
% End of journal.tex
%------------------------------------------------------------------------------
